\PassOptionsToPackage{unicode=true}{hyperref} % options for packages loaded elsewhere
\PassOptionsToPackage{hyphens}{url}
%
\documentclass[]{article}
\usepackage{lmodern}
\usepackage{amssymb,amsmath}
\usepackage{ifxetex,ifluatex}
\usepackage{fixltx2e} % provides \textsubscript
\ifnum 0\ifxetex 1\fi\ifluatex 1\fi=0 % if pdftex
  \usepackage[T1]{fontenc}
  \usepackage[utf8]{inputenc}
  \usepackage{textcomp} % provides euro and other symbols
\else % if luatex or xelatex
  \usepackage{unicode-math}
  \defaultfontfeatures{Ligatures=TeX,Scale=MatchLowercase}
\fi
% use upquote if available, for straight quotes in verbatim environments
\IfFileExists{upquote.sty}{\usepackage{upquote}}{}
% use microtype if available
\IfFileExists{microtype.sty}{%
\usepackage[]{microtype}
\UseMicrotypeSet[protrusion]{basicmath} % disable protrusion for tt fonts
}{}
\IfFileExists{parskip.sty}{%
\usepackage{parskip}
}{% else
\setlength{\parindent}{0pt}
\setlength{\parskip}{6pt plus 2pt minus 1pt}
}
\usepackage{hyperref}
\hypersetup{
            pdftitle={Lesson 14: Inference for Several Means (ANOVA)},
            pdfauthor={Homework},
            pdfborder={0 0 0},
            breaklinks=true}
\urlstyle{same}  % don't use monospace font for urls
\usepackage[margin=1in]{geometry}
\usepackage{graphicx,grffile}
\makeatletter
\def\maxwidth{\ifdim\Gin@nat@width>\linewidth\linewidth\else\Gin@nat@width\fi}
\def\maxheight{\ifdim\Gin@nat@height>\textheight\textheight\else\Gin@nat@height\fi}
\makeatother
% Scale images if necessary, so that they will not overflow the page
% margins by default, and it is still possible to overwrite the defaults
% using explicit options in \includegraphics[width, height, ...]{}
\setkeys{Gin}{width=\maxwidth,height=\maxheight,keepaspectratio}
\setlength{\emergencystretch}{3em}  % prevent overfull lines
\providecommand{\tightlist}{%
  \setlength{\itemsep}{0pt}\setlength{\parskip}{0pt}}
\setcounter{secnumdepth}{0}
% Redefines (sub)paragraphs to behave more like sections
\ifx\paragraph\undefined\else
\let\oldparagraph\paragraph
\renewcommand{\paragraph}[1]{\oldparagraph{#1}\mbox{}}
\fi
\ifx\subparagraph\undefined\else
\let\oldsubparagraph\subparagraph
\renewcommand{\subparagraph}[1]{\oldsubparagraph{#1}\mbox{}}
\fi

% set default figure placement to htbp
\makeatletter
\def\fps@figure{htbp}
\makeatother


\title{Lesson 14: Inference for Several Means (ANOVA)}
\author{Homework}
\date{}

\begin{document}
\maketitle

\textbf{Instructions: You are encouraged to collaborate with other
students on the homework, but it is important that you do your own work.
Before working with someone else on the assignment, you should attempt
each problem on your own.}

\begin{enumerate}
\def\labelenumi{\arabic{enumi}.}
\item
  In your own words, describe what an ANOVA test is used for.
\item
  What are three differences between an F-distribution and a
  t-distribution?
\end{enumerate}

It is very difficult and expensive to measure the protein requirement in
humans, but research into this area is very important. For example, how
much protein should you give to a patient in a health care facility who
must be fed enterally (i.e., through a feeding tube)? There are several
ways in which nutritionists have tried to measure the protein
requirement. Traditionally, they have used a method called Nitrogen
Balance.

In a nitrogen balance experiment, researchers provide a carefully
controlled diet containing prescribed amounts of protein to each subject
for an extended period of time. They then collect data on the amount of
protein utilized by the body. This includes collecting and analyzing
samples of urine, feces, blood, sweat, tears, exfoliated skin, etc. Most
researchers collect urine and fecal samples and estimate other losses.
The protein requirement is estimated as the level of intake required so
that the amount of protein consumed is exactly equal to the losses.
Because of the difficulty of measuring protein losses, and since protein
is essentially the only source for dietary nitrogen, nitrogen is used as
a marker for protein. A nitrogen balance experiment was conducted to
determine if there is a difference in the mean protein requirement of
individuals in four groups:

\begin{verbatim}
I. Old men (age 63-81)
II. Old women (age 63-81) 
III. Young men (age 21-46)
IV. Young women (age 21-46)
\end{verbatim}

Subjects were provided with a controlled diet for three months and were
required to comply with study protocol. The data set
\href{http://statistics.byuimath.com/index.php?title=Data}{ProteinRequirement}
gives the measured protein requirements for each of the subjects. Use
this information to answer questions 3 through 12.

\begin{enumerate}
\def\labelenumi{\arabic{enumi}.}
\setcounter{enumi}{2}
\item
  Create histograms to illustrate the protein requirements of the
  subjects within each group.
\item
  Give the appropriate summary statistics for each of the groups.
\item
  What type of test will be performed to compare the mean protein
  requirements of these four groups?
\item
  Are the requirements for this test satisfied? What requirements did
  you check? What are your conclusions?
\end{enumerate}

Conduct an ANOVA test using this data. (If the requirements from
Question 6 were not satisfied, conduct the test anyway. If you are
concerned about normality, there are good heuristic arguments suggesting
that protein requirements are normally distributed.)

\begin{enumerate}
\def\labelenumi{\arabic{enumi}.}
\setcounter{enumi}{6}
\item
  Write the appropriate null and alternative hypotheses. Use
  \(\alpha = 0.05\).
\item
  ind the test statistic and its value. Also give both degrees of
  freedom.
\item
  Give the P-value.
\item
  Give the decision rule for this test.
\item
  Based on the decision rule, what do you conclude?
\end{enumerate}

Conjugated linoleic acid (CLA) is found in milk fat from cows. It has
recently been discovered that CLA has several health-promoting
characteristics, including cancer risk reduction. Researchers in
Alberta, Canada wanted to know if supplementing the diet of cattle with
monensin (an antibiotic), safflower oil, or both would affect the amount
of CLA in the milk fat (measured in percent). Seven cattle were randomly
assigned to each of the following diets:

\begin{verbatim}
I. Control: diet was not supplemented with monensin or safflower oil,
II. MON: diet was supplemented with monensin,
III. SAFF: diet was supplemented with safflower oil
IV. SAFF/M: diet was supplemented with monensin and safflower oil
\end{verbatim}

After two weeks, the CLA content in the milk fat from each cow was
analyzed. The results are summarized in the data file
\href{http://statistics.byuimath.com/index.php?title=Data}{ConjugatedLinoleicAcid}.
Use this information to answer questions 13 through 22.

\begin{enumerate}
\def\labelenumi{\arabic{enumi}.}
\setcounter{enumi}{11}
\item
  Create histograms to illustrate the CLA content in milk fat for the
  four treatment groups. What do you observe? Without performing a
  hypothesis test, do you think there is a difference in the mean CLA
  content in milk fat for the four treatment groups? Justify your
  answer.
\item
  Give the appropriate summary statistics for each of the groups.
\item
  Are the requirements for an ANOVA test satisfied? What requirements
  did you check? What are your conclusions?
\end{enumerate}

Conduct an ANOVA test using this data. (If the requirements from
Question 15 were not satisfied, conduct the test anyway.)

\begin{enumerate}
\def\labelenumi{\arabic{enumi}.}
\setcounter{enumi}{14}
\item
  Write the appropriate null and alternative hypotheses. Use α = 0.05.
\item
  Find the test statistic and its value. Also give both degrees of
  freedom.
\item
  Give the P-value.
\item
  Give the decision rule for this test.
\item
  Based on the decision rule, what do you conclude?
\item
  If you were a researcher overseeing this study, what action would you
  recommend based on these results?
\end{enumerate}

\end{document}
